\hypertarget{general-information-scripts-commands-and-syntax}{%
\section{\texorpdfstring{\hfill\break
General Information: Scripts, Commands and
Syntax}{ General Information: Scripts, Commands and Syntax}}\label{general-information-scripts-commands-and-syntax}}

\hypertarget{types-of-scripts}{%
\subsection{Types of scripts}\label{types-of-scripts}}

\hypertarget{local-scripts}{%
\subsubsection{Local scripts}\label{local-scripts}}

Any script that is running on an object or Actor in the game (assigned
in the script-dropdown field of the object or Actors object window) is a
local script. Local scripts are only active if the cell is loaded --
this is the current interior cell, or the current and all directly
neighboring exterior cells. When the object is outside of this range the
script is not running, but the local variables are saved. You cannot
stop a local script using Stopscript.

\hypertarget{global-scripts}{%
\subsubsection{Global scripts}\label{global-scripts}}

Any script that is not attached to any object is a global script, and is
by default not executed until you call it (see below). Note that there
is no default object for a global script to work on, so objects must
always be specified: while the following will work in a local script
attached to an NPC:

\begin{lstlisting}
	AITravel 1150, 8899, 1110
\end{lstlisting}

You will have to specify the NPC in a global script:

\begin{lstlisting}
	"NPC_ID"-> AITravel 1150, 8899, 1110 ;NPC_ID is the ID, the
	unique identifier for
	
	;each object in the editor
\end{lstlisting}

Global scripts are active all the time once they have been activated and
until they are specifically terminated. Thus, once activated, they will
be processed every frame as described for locals scripts above. That is
why they should be used with caution, as too many, or too complicated
global scripts can easily slow the game down a lot.

The command to start a non-active script is:

\begin{lstlisting}
	StartScript, "Script ID"
\end{lstlisting}

With Tribunal and Bloodmoon you also have the option to make a script
start automatically with the game. In the TESCS under the menu
Gameplay/Edit starting Scripts/ you can add any script to the list of
automatically started scripts.

The function to terminate a global script is:

\begin{lstlisting}
StopScript, "Script ID"
\end{lstlisting}

Variables local to a global script will be saved temporarily when the
script is stopped and restarted. In order to ensure that variables are
always saved, the script must be run at least once every load session.
If you need to be sure the variables are reset, you should reset them
yourself: don't rely on it happening automatically.

A simple way to ensure variables are saved is to use a startscript (only
available with expansions); so for example the startscript might look
something like this:

\lstinputlisting{scripts/KeepVarsScript.txt}

...and the script "MyScript" might look like this:

\lstinputlisting{scripts/MyScript.txt}

It is possible to use the StartScript function to run global scripts
that are tied to an object or Actor. These are called "targeted
scripts".

\begin{lstlisting}
	"Object_ID"-> StartScript "Script_ ID"
\end{lstlisting}

These scripts resemble both local scripts (in that the functions called
always default to the object or Actor the script targets) and global
scripts (in that they are always running and can be terminated with
StopScript).

\textbf{Note:} read more on the special case of "targeted scripts" in
the Tips and Tricks section.

\hypertarget{syntax}{%
\subsection{\texorpdfstring{\hfill\break
Syntax}{ Syntax}}\label{syntax}}

There are some unique features in TES script that should be explained
before we go into more detail.

\hypertarget{beginning-and-ending-scripts}{%
\subsubsection{Beginning and ending
scripts}\label{beginning-and-ending-scripts}}

\begin{lstlisting}
	Begin Script_ID
	
	End
	
	End Script_ID
\end{lstlisting}

Every script must have the Begin and End tags. The specified name will
also be the ID you will reference the script by (be it from other
scripts or inside the TES CS object windows). A script may start out
with comments, but the first line of real code must be "Begin
xxxxxxxxx".

As with other objects, it is recommended that you give your scripts a
unique tag. I usually use GBG\_Scriptname. This ensures that your
scripts are easily identified, they all are neatly listed in one block
in the scripts list, and there is little danger of conflicts with other
mods using the same name. Using leading underscores e.g. \_Scriptname is
not recommended.

If you want your scripts to appear at the top of the list of scripts,
put a 1 in front, for example, 1YAC\_ScriptName

\hypertarget{general-syntax-for-functions}{%
\subsubsection{\texorpdfstring{General syntax for functions:
}{General syntax for functions: }}\label{general-syntax-for-functions}}

Functions are not case sensitive, but using the case sensitive form as
suggested by Bethesda (e.g. GetSpellEffects instead of getspelleffects)
makes scripts easier to read, so it's recommended to keep it that way.

\begin{lstlisting}
	"Object_ID"-> Function, [parameters]
\end{lstlisting}

This is the format for all functions that act on or refer to a
specifiable object in the game world. The "arrow" or "fix" designates
which object a function will be performed on. The "object\_ID" is the
unique ID that is given to each object in the editor (usually the first
field in any object window). You need this ID, not the Name! If you call
a function without a designated Object-ID, the function will be
performed on the default object, which is the object the script is
attached to.

This is not to say that there might not be another object referenced as
a parameter:

\begin{lstlisting}
	"Object_ID1"-> Function, "Object_ID2"
\end{lstlisting}

\textbf{Notes:}

Functions with a "fix" will only compile if the object has already been
placed into the game world in the editor (with at least one reference),
otherwise the script will not compile. If you compile the script and
then delete all references, the game will give errors on load and CTD.

More than one function may be used in a \emph{set} expression, but the
functions always apply to the same reference. For example:

\begin{lstlisting}
	set SomeVar to ( player-> GetStrength ) + (
	player-> GetEndurance )
\end{lstlisting}

and

\begin{lstlisting}
	set SomeVar to ( player-> GetStrength ) + ( GetEndurance )
\end{lstlisting}

will both have the same effect, and specifying different references will
not work.

Referencing non-unique items with the "fix" will usually perform the
function only on the first reference of the object! So using this in a
global script:

\begin{lstlisting}
	"cliff racer"-> ModCurrentHealth -1000
\end{lstlisting}

will not have the desired effect, it will only kill one of the annoying
bastards. However attaching a script to the creature with just

\begin{lstlisting}
	ModCurrentHealth -1000
\end{lstlisting}

Would do the trick, because every reference of the cliff racer will have
the script running, and apply the function to itself. Note that this
does not apply to all functions (e.g. SetHealth in the above example
would affect all references the player has not yet encountered).

A number of functions refer only to the player or not to an object at
all, and are therefore using the fix is meaningless or may produce
errors. E.g.:

\begin{lstlisting}
	If ( GetPCRank == 0 )
	
	If ( CellChanged == 0 )
	
	FadeOut, 2
\end{lstlisting}

\hypertarget{general-syntax-commas-parentheses-and-spaces}{%
\subsubsection{General syntax: Commas, parentheses and
spaces}\label{general-syntax-commas-parentheses-and-spaces}}

TES Script is not too picky about syntax. Case mostly doesn't matter,
commas can be left out, and spaces are mostly ignored. Nevertheless I
would advise adhering to the following principles:

\begin{itemize}
\item
  Either avoid commas, or always use commas: inconsistent usage can
  cause problems.
\item
  If the ID contains a space or begins with an underscore, you must use
  quotation marks: "Object ID" or "\_Object\_ID". Better to avoid spaces
  altogether: Object\_ID
\item
  Get used to \textbf{always} leaving spaces around parentheses and
  operators, sometimes it seems to cause problems if you don't: if (
  variable == 1 ), not: if (variable==1).\\
  While this doesn't matter most of the time it generates weird and
  almost untraceable errors sometimes, so you are much better off always
  leaving a space.
\item
  The fix (->) is a little more complicated. If IDs are
  contained in quotes, you should \textbf{not} leave spaces around the
  fix:
\end{itemize}

\begin{quote}
"Sirollus Saccus"-> GetItemCount netch\_leather\_greaves
\end{quote}

The above will work, but spaces around the fix would cause problems.
(Thanks to Simpleton and DinkumThinkum for this info.) However, since it
has also been reported that a \emph{lack} of quotation marks can cause
problems in combination with spaces round the fix, I'm going to
recommend that you don't leave spaces round the fix at all (should be
fine in all cases).

\hypertarget{comments}{%
\subsubsection{Comments}\label{comments}}

Comments are marked by a semicolon ;. Comments can be added in their own
lines or behind lines with code.

% \textbf{ here
\begin{lstlisting}
	; enter sneak mode
	
	Fargoth-> ForceSneak
	
	Fargoth-> AiTravel -11468.595,-71511.531,173.728
	;goes to tree
\end{lstlisting}

\hypertarget{indentation-using-tabstops}{%
\subsubsection{Indentation / using
tabstops}\label{indentation-using-tabstops}}

For your own sake, use proper indentation (tabstops) for if-elseif
constructs -- makes it much easier to keep track of them, so you don't
forget an endif at the end.

% In the tips and tricks section you will find a link to an external EMACS editor mode for TES script that will do automatic indentation.

\begin{lstlisting}
	If ( variable1 )
	
	If ( variable2 )
	
	[do something]
	
	endif
	
	endif
\end{lstlisting}
	
is better than

\begin{lstlisting}
	If ( variable1 )
	
	If ( variable2 )
	
	[do something]
	
	endif
	
	endif
\end{lstlisting}

\hypertarget{variables}{%
\subsection{Variables}\label{variables}}

\hypertarget{types-of-variables}{%
\subsubsection{Types of variables}\label{types-of-variables}}

There are three types of variables in the TES script language: short,
long and float. According to the manual these cover the following data
ranges:

	Short -32,768 to 32,767 (signed integer)
	
	Long -2,147,483,648 to 2,147,483,647 ( long signed integer)
	
	Float 3.4E +/- 38 (float, 7 digits)

Apparently the boundaries for Long given here are only partly correct,
in the TES-CS you can assign a maximum value of 2147483520 (Forum info /
Argent).

Theoretically there should be string variables too, but to my knowledge
these are not implemented. Unfortunately there are also no data types to
store Object\_Id's, which limits the power of the scripting language to
a certain extent.

Variables can be grouped into \emph{local} variables (valid only in the
script that declares them) and \emph{global} variables (valid in every
script).

\textbf{Note}: A long global is effectively a float! In a script, local
to that script a long will have a full 32 bits of precision. But used as
a global, the number of bits of precision drops to 24, as when it exists
globally a long \emph{is} a float. Mental Elf discovered this when
attempting to use "bit packing" to get 32 flags into a global long
(forum info / mental elf).

\textbf{A Note about Floats:}

Floating point numbers in a computer are stored in a format similar to
that used for scientific or engineering notation. The number is rounded
off to a fixed number of significant digits, any leading or trailing
zeros are dropped, and an exponent is added to indicate the correct
location of the decimal or binary point.

The number 2385901045 would be stored similar to this:

2.385901x10\textsuperscript{9}

In fitting the number into a fixed number of bytes, we have lost the end
3 digits, so the number is no longer quite what you set it to.

If you set a float to a small number, like 5.5, it would still be stored
as 5.5, because as it doesn't take up much room, no numbers need to be
cut off.\\
\strut \\
The advantage of this format is that a wide range of values can be
expressed using a relatively small number of digits or bytes. The
disadvantage is that the rounding off means floating point numbers don't
have the precision of an integer, so they shouldn't be used when you
need to make exact comparisons or keep track of exact counts.\\
\strut \\
For example, a floating point variable is fine if you just need to check
if a distance is greater or less than a certain value. But if you need
to check if the player has specific number of some item in their
inventory, then you want to use a long or short integer variable for the
count.

(Forum info / DinkumThinkum)

An example by BungaDunga shows that when doing math with floats, the
answer isn't always what you expect:

% \textbf{ here
\begin{lstlisting}
	Float num1
	
	Float num2
	
	Float num3
	
	Set num1 to ( 1 / 3 )
	
	;num1 is now 0.3333333
	
	Set num2 to 3
	
	;num2 is now 3
		
		set num3 to num1 * num2}
	
	;num3 is now 0.9999999, rather than what you would expect, 1.
		
		if ( num3 == 1 )
		; Never happens.
		endif
\end{lstlisting}

A lot of functions return float values (e.g. GetDistance, GetScale,
GetSecondsPassed\ldots): the same thing applies. Test a range, not an
exact value! E.g.:

\begin{lstlisting}
	if ( ( GetPos x ) == 500 )
	
	;will be false if it's 499.9999, or 500.0001, etc
	
	endif
\end{lstlisting}
	
but this will work:

\begin{lstlisting}	
	if ( ( GetPos x ) > 499.5 )
	
	if ( ( GetPos x ) < 500.5 )
	
	;it's close enough for me
	
	endif
	
	endif
\end{lstlisting}

\protect\hypertarget{_Toc182634499}{}{}

Another common mistake is to check if the GameHour is an exact number.
As the Gamehour variable is increased every frame, and every frame is a
slightly diffrent length, the Gamehour variable soon ends up with a
value like this, 10.12853. So never test if the gamehour is exatly equal
to a value, as it is very unlikley to happen, test if it is greater than
or equal to a value.

\hypertarget{local-variables}{%
\subsubsection{Local variables}\label{local-variables}}

Local variables these have to be declared in the script:

	Float floatvarname
	
	Short shortvarname
	
	Long longvarname

Local variables are unique to a specific instance of a local script.
This means that the local variables in multiple objects with the same
script do not influence each other. The names you use for variables are
pretty much up to you as long as they start with a letter, but you have
to avoid using function names (this will result in errors during
runtime) and reserved characters (e.g. - + / * = " )( etc.) which will
result in compiler errors. E.g. "variable-1" will not work as a variable
name. Underscores as in "my\_variable" are ok, but avoid leading
underscores. A dot has a reserved meaning as well (see
"\emph{Referencing variables in other local scripts}" below).

\hypertarget{global-variables}{%
\subsubsection{Global variables}\label{global-variables}}

To declare a global variable go to the Gameplay menu and select Globals.
Right click for "new", name it, and set the type and starting value for
the new global variable, if one is needed. By default it will be 0.
Global variables are very useful for involved quests when you need to
keep track of things over an extended time and space. They are also a
simple way to share information between different scripts

\textbf{Note}: if you declare a local variable with the same name as a
global variable, the global variable will become invisible for this
script. Do NOT declare a global variable as a local!

\hypertarget{referencing-variables-on-other-objects-and-scripts}{%
\subsubsection{Referencing variables on other objects and
scripts}\label{referencing-variables-on-other-objects-and-scripts}}

Set \ldots{} to

If a unique object has a local script running on it you can change
variables from outside the script in the following way:

\begin{lstlisting}
	Set MyObject.variable to 100
\end{lstlisting}

or

\begin{lstlisting}
	Set MyObject.variable to local_variable
\end{lstlisting}

This method changes a local variable in the object's script. The object
must have a script on it for this to be valid. The object does not have
to be in an active cell when setting the variable, but it will only work
if the cell containing the target object/(script) has previously loaded
-(Cyran0). \textbf{Note:} The scripting system looks at the first object
in the database, thus you should only reference objects that are unique
(exist only once).

Note that the reverse does \textbf{not} work:

\begin{lstlisting}
	Set local_variable to MyObject.variable ;this doesn't work!
\end{lstlisting}

Use a global variable to transfer information in this way, or set
local\_variable from the \textbf{other} script using the syntax above.

\begin{lstlisting}
	if ( anotherobject.x > 0 )
\end{lstlisting}

apparently works.

Furthermore I realized only recently that this syntax also works for
global scripts:\\
\strut \\

\begin{lstlisting}
	set Global_script_name.variable to 1
\end{lstlisting}

\strut \\
This is useful to avoid using more global variables then necessary or
also as a console command to debug global scripts.

\textbf{Note:} If your object or global script starts with a number, you
need to put quotes around it.

\begin{lstlisting}
	"11NPC01".RemoteVar ;-\/-\/-Good\\
	"11NPC01.RemoteVar" ;-\/-\/-BAD
	
	11NPC01.RemoteVar ;-\/-\/-BAD
\end{lstlisting}

A caution when setting variables from outside the script
(DinkumThinkum):

\emph{If you recompile the target script (i.e., the local script on
'MyObject'), it's a good idea to also recompile any scripts that
reference variables in that script. Reason: if the target 'variable'
winds up in a different position in the target script's list of
variables, then any scripts trying to set that variable will break if
they're not recompiled.}

I.e., if 'variable' is the 14th variable declared in the local script on
'MyObject', and you add a new variable ahead of it so that 'variable' is
now the 15th variable declared, other scripts will need to be recompiled
in order to find 'variable' in its new position, otherwise the script
will end up changing the wrong variable, leading to very strange bugs.

\emph{One way to reduce the chances of this tripping you up is declare
variables referenced by other scripts first,} before \emph{other
variables that aren't referenced externally. Then just be careful to
only add or remove variables} after \emph{the block of externally
referenced variables. However (to be safe), it's still a good idea to
recompile scripts that reference variables in other scripts any time
you've added or removed variables in the other scripts.}

\hypertarget{using-variables-in-functions}{%
\subsubsection{Using variables in
functions}\label{using-variables-in-functions}}

Unfortunately it is one of the limitations of TES script that only
certain functions accept variables as parameters, which poses some
definite limits. The type of arguments functions accept is indicated in
the list of functions below.

\textbf{Note:} For some functions where both a Get-Function and a
Set-Function exist, a workaround can be constructed by using the while
function (see below).

%\hypertarget{section}{%
%\subsubsection{}\label{section}}

\hypertarget{operators-mathematical-calculations}{%
\subsubsection{\texorpdfstring{Operators / mathematical calculations
}{Operators / mathematical calculations }}\label{operators-mathematical-calculations}}

You can use the standard operators to do calculations in scripts:

\begin{longtable}[]{@{}
  >{\raggedright\arraybackslash}p{(\columnwidth - 2\tabcolsep) * \real{0.82}}
  >{\raggedright\arraybackslash}p{(\columnwidth - 2\tabcolsep) * \real{0.18}}@{}}
\toprule
\begin{minipage}[b]{\linewidth}\raggedright
Addition:
\end{minipage} & \begin{minipage}[b]{\linewidth}\raggedright
+
\end{minipage} \\
\midrule
\endhead
Subtraction: & - \\
Multiplication: & * \\
Division: & / \\
\bottomrule
\end{longtable}

The syntax is as follows:

Set result\_var to (var\_a + var\_b)

Instead of variables, literal values are also allowed. I assume that
standard operator precedence applies ( * and / are calculated before +/-
). Since it has not been properly tested I usually always use
parentheses to be on the safe side. You can use parentheses according to
standard mathematical rules:

\begin{lstlisting}
	set ln to ( ln + ( k10 * math_ln10 ) + ( k2 * math_ln2 ) )
\end{lstlisting}

A warning: There are different opinions on the forums on the use of
several operators in one line. Some people report a lot of problems with
this, I myself have successfully used at least four operators and
variables in a single line. There appears to be an issue with very long
additions (e.g. adding up more than 20 variables in one line of code)
that causes a mod to crash the game upon loading. If this happens, split
the calculations to several lines.

\hypertarget{notes-by-dinkumthinkum}{%
\subparagraph{Notes by DinkumThinkum:}\label{notes-by-dinkumthinkum}}

\hypertarget{eleven-variables-in-a-single-set-statement-will-cause-the-following-error-when-loading-the-game-followed-by-a-crash-to-desktop}{%
\subparagraph{\texorpdfstring{\emph{Eleven variables in a single set
statement will cause the following error when loading the game, followed
by a crash to
desktop:}}{Eleven variables in a single set statement will cause the following error when loading the game, followed by a crash to desktop:}}\label{eleven-variables-in-a-single-set-statement-will-cause-the-following-error-when-loading-the-game-followed-by-a-crash-to-desktop}}

\hypertarget{need-more-room-for-zero-pointers-in-scriptreplaceglobalsindata}{%
\subparagraph{"Need more room for zero pointers in
Script::ReplaceGlobalsInData"}\label{need-more-room-for-zero-pointers-in-scriptreplaceglobalsindata}}

as soon as you click the button to acknowledge the error, CTD. (That
script name wasn't any script of mine; it's something internal to the
game.)

Six variables in a Set statement work fine. Don't know the maximum
number, but it's obviously at least six and definitely less than eleven.

There isn't much in the way of dedicated mathematical functions in TES
script. There is the 'Random' function and Tribunal added the
'GetSquareRoot' function (see below). If you need more complex
functions, I suggest downloading Soralis' Math Mod (available from
Planet Elder Scrolls). It's a collection of scripts that allow you to do
complex calculations. Here is a short excerpt from the readme to give
you an idea:

\emph{"This mod adds the ability to use various math functions within
Morrowind's scripts.}

Specifically, these are the scripts that are added:"

\begin{longtable}[]{@{}
  >{\raggedright\arraybackslash}p{(\columnwidth - 8\tabcolsep) * \real{0.20}}
  >{\raggedright\arraybackslash}p{(\columnwidth - 8\tabcolsep) * \real{0.15}}
  >{\raggedright\arraybackslash}p{(\columnwidth - 8\tabcolsep) * \real{0.22}}
  >{\raggedright\arraybackslash}p{(\columnwidth - 8\tabcolsep) * \real{0.32}}
  >{\raggedright\arraybackslash}p{(\columnwidth - 8\tabcolsep) * \real{0.12}}@{}}
\toprule
\endhead
Name & Check/Done & Inputs & Outputs & Accuracy \\
MathScripts & N/A & N/A & N/A & N/A \\
MathConstants & N/A & N/A & N/A & N/A \\
SquareRoot & 1 & math\_sqrt & math\_result, math\_imag & 7 \\
SineScript & 2 & math\_angle & math\_sin, math\_cos, math\_tan & 7 \\
ArcsineScript & 3 & math\_arc & math\_sin, math\_cos & 6-7 \\
NaturalLog & 4 & math\_log & math\_result, math\_imag & 4-5 \\
LogScript & 5 & math\_log, math\_base & math\_result, math\_imag &
3-4 \\
intPower & 6 & math\_value, math\_power & math\_result & 7 \\
intRoot & 7 & math\_value, math\_root & math\_result, math\_imag &
6-7 \\
Modulus & 8 & math\_value, math\_mod & math\_result & 6-7 \\
Antiln & 9 & math\_log & math\_result & 4-5 \\
Antilog & 10 & math\_log, math\_base & math\_result & 2-3 \\
AbsoluteValue & 11* & math\_abs & math\_abs & 7 \\
PowerScript & 12 & math\_value, math\_power & math\_result, math\_imag &
2-3 \\
\bottomrule
\end{longtable}

Unfortunately many of these functions are rather slow, and not suited
for real time calculations. For sine and cosine in particular, check out
JDGBOLT's script in the tips and tricks section, which uses
pre-calculated values for a very rapid calculation. You can also find
more information on UESP Wiki.

MWSE (Morrowind Script Extender) also adds trig functions to Morrowind.

%\protect\hypertarget{_Toc182634504}{}{}

\hypertarget{testing-conditions}{%
\subsection{Testing conditions}\label{testing-conditions}}

\hypertarget{use-of-if-elseif-conditions}{%
\subsubsection{Use of if\ldots{} elseif
conditions}\label{use-of-if-elseif-conditions}}

	If (condition)
	
	Elseif (condition)
	
	Else
	
	Endif

Much of TES scripting relies on the use of \emph{if\ldots{} elseif}
conditions and variations thereof. It is very important to fully
understand these, and the conditions that can be used in them to test
conditions of the game world to trigger events.

In general a condition is "true" when it has the value of 1 (or simply
"not 0" (e.g. -1 is "true" in TES Script too) and "false" when the value
is 0. There are thus certain functions in the game that return "true"
under certain conditions. For example see the GetAIPackageDone function
further down. It returns "true" (= 1) for one frame when an AIPackage
has finished. All lines of code between the if-statement and endif
(which concludes a so-called "if block") will be executed only when the
condition is true.

Thus:

% \textbf{ here
\begin{lstlisting}
	If ( GetAIPackageDone )
	
	;[do something here]
	
	endif
\end{lstlisting}

will be only executed when the GetAIPackageDone function currently
returns the value 1.

In addition to just a value, a condition can also be a more explicit
test of conditions. The conditions, such as "A equals B" or "A is
greater than B" will be evaluated, and the whole expression will be true
(equals 1) when the condition is fulfilled. E.g.:

% \textbf{ here
\begin{lstlisting}
	if ( GetAIPackageDone == 1 )
	
	;[do something here]
	
	endif
\end{lstlisting}

Is equivalent to the example above. This second syntax tests a condition
and returns "true" if the condition is met. You can check for the
following conditions:

% orginally in ""
\index{Equal to}
\index{Greater than or equal to}
\index{Greater than}
\index{Smaller than or equal to}
\index{Smaller than}
\index{Unequal to}

\begin{longtable}[]{@{}
  >{\raggedright\arraybackslash}p{(\columnwidth - 2\tabcolsep) * \real{0.54}}
  >{\raggedright\arraybackslash}p{(\columnwidth - 2\tabcolsep) * \real{0.46}}@{}}
\toprule
\endhead
\textbf{Verbose} & \textbf{Conditional operator} \\
"Equal to": & == \\
"not equal to" & != \\
"Greater than" & > \\
"Smaller than" & < \\
"Greater than or equal to" & >= \\
"Smaller than or equal to" & <= \\
\bottomrule
\end{longtable}

If you have several independent if blocks each will be evaluated
separately. If, e.g. the following if blocks are in your script:

\begin{lstlisting}
	if ( timer >= 2 )
	
	;[Block A,do something here]
	
	endif
	
	if ( timer <= 3 )
	
	;[Block B, do something else here]
	
	endif
\end{lstlisting}

Both will be true and the code in the blocks will be executed when timer
is between 2 and 3.

You can nest if-blocks if you want to make sure that two conditions are
fulfilled simultaneously:

\begin{lstlisting}
	if ( timer >= 2 )
	
	if ( controlvar == 0
	
	;[do something here]
	
	endif
	
	endif
\end{lstlisting}

The code in the inner if block will only be executed when both
conditions (timer >= 2 and controlvar == 0 ) are fulfilled at
the same time.

For more elaborate constructions you can use Else and Elseif. Elseif
tests for a separate condition if the preceding condition has failed
(but not if the previous condition was fulfilled)

\begin{lstlisting}
	if ( timer >= 2 )
	
	;[Block A,do something here]
	
	elseif ( timer <= 3 )
	
	; [Block B, do something else here]
	
	endif
\end{lstlisting}

unlike the example above, both blocks can not be true at the same time
now. Either timer is >= 2, then block A gets executed, or
timer is not >= 2 but <= 3, meaning effectively timer
is < 2, then block B gets executed. In all other cases,
neither block will be executed. You can have several elseifs behind each
other:

\begin{lstlisting}
	if ( counter == 1 )
	
	;[Block A,do something here]
	
	elseif ( counter == 2 )
	
	; [Block B, do something else here]
	
	elseif ( counter == 3 )
	
	; [Block C, do something else here]
	
	endif
\end{lstlisting}

An else creates a "default condition". The code following else will be
executed if all previously tested conditions are false:

% \textbf{ here
\begin{lstlisting}
	If ( foo_var == 1 )
	
	[Block A: do something]
	
	elseif ( foo_var == 2 )
	
	[Block B: do something else]
	
	else
	
	[Block C: do something completely different]
	
	endif
\end{lstlisting}

In this example Block C will be executed it foo\_var is neither 1 nor 2.

\hypertarget{notes-and-cautions}{%
\subparagraph{Notes and cautions:}\label{notes-and-cautions}}

In my experience it is safest to use \emph{elseif} instead of multiple
separate \emph{if} statements if you test different states of one
variable.

Be careful using a function in a conditional ('If' statement); simple
functions seem to be OK, but ones with numeric parameters don't seem to
be reliable.

% TODO: clean up for easier reading
For example:\\
"If ( NewType != OldType )" worked correctly, but "If ( (
Player-> GetArmorType 0 ) != OldType )" always registered as
'Not equal', even if the variable 'OldType' had the same value as the
armor type that should have been returned by 'Get ArmorType 0'
function.\\
So use "Set NewType to ( Player-> GetArmorType 0 )", followed
by a separate 'If' statement to compare 'NewType' to 'OldType'.
-(DinkumThinkum)

\hypertarget{while-conditions}{%
\subsubsection{While conditions}\label{while-conditions}}

	While ( condition )
	
	; things to do
	
	EndWhile

The while command differs from the if command in that it is repeated
within one frame until the condition is fulfilled. This is best
explained with an example:

% \textbf{ here
\begin{lstlisting}
	Short desiredAmnt
	
	SetStrength 0
	
	while ( GetStrength < desiredAmnt ) ; non-literal
	value to match
	
	modStrength 1
	
	endwhile
\end{lstlisting}

This will set strength to the value in variable desiredAmnt after one
frame. The following script however would need an undetermined amount of
frames to do this, because the if condition is only called once each
frame:

\begin{lstlisting}
	if(getStrength < desiredAmnt) ; non-literal value to match
	
	modStrength 1
	
	endif
\end{lstlisting}

On the other hand the first example can potentially cause "freezing" (if
the value would be very high) while the second won't.

Note that this is a workaround for some functions to the problem of
functions not accepting non-literal values (variables) as arguments.

\hypertarget{constructing-boolean-operations-in-tes-script}{%
\subsubsection{Constructing Boolean operations in TES
Script}\label{constructing-boolean-operations-in-tes-script}}

Unfortunately there are no Boolean operators (AND, OR, NOT, XOR, \ldots)
in the scripting language. Thus you need to construct these yourself
using if\ldots{} elseif structures.

Instead of AND:

% \textbf{ here
\begin{lstlisting}
	if ( variable1 AND variable2 ); does not exist
	
	[do something]
	
	endif
\end{lstlisting}
	
you have to use:

\begin{lstlisting}	
	If ( variable1 )
	
	If ( variable2 )
	
	[do something]
	
	endif
	
	endif
\end{lstlisting}
	
For OR constructs:

\begin{lstlisting}	
	if ( variable1 OR variable2 )
	
	[do this]
	
	endif
\end{lstlisting}
	
you can use elseif constructions:

\begin{lstlisting}	
	If ( variable1 )
	
	[do this]
	
	elseif ( variable2 )
	
	[do this]
	
	endif
\end{lstlisting}