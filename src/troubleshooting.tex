% Chapter sectioning
\hypertarget{troubleshooting}{%
\section{Troubleshooting}\label{troubleshooting}}

\hypertarget{general-hints}{%
\subsection{General hints}\label{general-hints}}

\begin{itemize}
\item
  A good way to debug is to insert MessageBox commands at key points in
  your script.
\item
  If you produce error messages and can't find the cause, try to remove
  suspect lines of codes one by one (using ; to mark them as comments)
  to pinpoint the error causing line.
\item
  Pay attention to the error codes that the editor and the game give
  you, they usually let you pinpoint the source of the problem to some
  degree.
\item
  When you make changes to a script, it's a very good idea to save the
  script \textbf{twice} afterwards, to make sure the compiler properly
  checks it for errors.
\end{itemize}

For example, if you delete a variable declaration but forget to remove
the variable from the script, the compiler may not report an error the
first time you save the script. This also happens when you change the
name of a script (in the 'Begin' statement), but leave the original name
in a StopScript statement.

\begin{quote}
Apparently, the compiler doesn't update its internal list of variables,
script names, etc. until \textbf{after} it compiles the script. End
result is that the compiler will use the stored variable declarations
and script names from its previous pass through the script, rather than
the current ones.

This probably explains why you have to save a script at least once
before its name (from the 'Begin' statement) will be recognized in a
StopScript statement in the same script. -(DinkumThinkum)
\end{quote}

\hypertarget{the-console}{%
\subsection{The Console}\label{the-console}}

\hypertarget{using-the-console-to-check-variables}{%
\subsubsection{Using the Console to check
variables:}\label{using-the-console-to-check-variables}}

In the game you can use the console to check on the state of variables.
Bring up the console window (standard key is \textasciitilde{} or
whatever key is left of the "1" key if you are using a non-US keyboard
layout) and type "sv" -- this lists all the global variables with their
values, then all running global scripts (including targeted scripts)
with their variables. Now find an object that has a script running on
it. Bring up the console again, left click on the object -- the console
window title will change. Type "sv" again -- now the local variables for
the script running on this object will be listed. To check a single
variable, use "show \emph{var\_name}", however, if trying to get a var
from a object that doesn't have a script, it can sometimes crash the
game.

\hypertarget{using-the-console-to-quickly-test-scripts}{%
\subsubsection{Using the Console to quickly test
scripts:}\label{using-the-console-to-quickly-test-scripts}}

The console can help you test your scripts. For objects you don't need
to place them with the editor and then go there. Just write down the ID
for your object, load any savegame and then, in the console type:

PlaceatPC "My\_Object" 1,1,1

This will drop the object directly at your feet.

Player-\textgreater AddItem "My\_Object" 1

This will drop the item into your inventory

To go to a specific location use

coc "cell\_name"

for interior cells or

coe -1,-7

for exterior cells (write down the cell coordinates in the editor). coc
works for exterior cells too but since most exterior cells have
non-unique names you may not land quite where you want. It can still be
useful though, for example coc balmora will take you somewhere in
Balmora.

tcl

Toggles collision -- float through walls, visit difficult to reach
places easily.

Tgm

Toggle God mode -- test without worrying about some monster killing you.

Also, the invaluable console commands to run the game in "debug" mode
allow one to view hard numbers about spell effects, chances to hit, etc
(thanks to Wakim). Some of these are:

1) press the "\textasciitilde" key to call up the console.

2) type "tcs" into the console and hit return.

3) click (right or left click, can't recall) anywhere on the screen
outside of the console box to continue game time while leaving the box
displayed and active.

That's it. There are other variants on the "tcs" command, such as (but
not limited to) "tks" and "tms" which all stand for "toggle xxxxxx
statistics" where xxxxx is: c = combat, m = magic, k = kill, or what
have you.

\hypertarget{error-messages-malfunctions-and-common-causes}{%
\subsection{Error messages, malfunctions and common
causes}\label{error-messages-malfunctions-and-common-causes}}

\hypertarget{in-the-editor}{%
\subsubsection{In the editor}\label{in-the-editor}}

The editor usually indicates the line in which the problem was
encountered, so often errors reported here are relatively easy to fix.

"Mismatched If/else/endif starting on line..."

One or several if statements are not closed. Use tabs to facilitate
finding missing endifs. This also happens when wrong names are used for
variables functions or other typing errors.

"Function reference object "Foobject" not found"

This error indicates that the object in question e.g.

Foobject-\textgreater GetDistance Player

Does not exist. When you write a script and then create a new object to
be thus referenced you will also get this error. Close the editor window
and reopen it to register the object with the compiler.

"Could not find variable or function "Foobject""

Another syntax error, indicating you have used an undefined variable,
function or a non-existent object.

"Script command "foofunc" not found on line 3"

A command / function was not recognized by the compiler. Usually results
from typos.

"You need to end a script with script End scriptname"

Indicates that the compiler did not find the End command. This error can
also appear when a previous error disrupts the compilation process.

"Syntax error Line 20. Miss matched parenthesis"

Indicates that you do not close all parentheses you opened or vice
versa.

"You need to enter a value on line 20"

You have not supplied all the arguments needed by a specific function.

\hypertarget{in-game-error-messages}{%
\subsubsection{In game error messages:}\label{in-game-error-messages}}

"EXPRESSION in script..."

Followed by

"RightEval ..."

This error indicates that Variables are not declared. Most often this
happens with game variables that are almost used like functions, e.g.
OnPCEquip.

EXPRESSION in general appears mostly when variables have not been
declared in the script.

"LeftEval"

This error seems to come up when you have accidentally declared a
Function as a variable. The following lines e.g. would produce this
error:

short ScriptRunning

if (ScriptRunning "MyScript" == 1 )

Both of the above can also be caused by not having spaces at the proper
places. Always leave a space between parentheses and function calls,
variables etc.

If ( OnActivate == 1 ), not if (OnActivate==1). This only causes errors
very rarely, but it sometimes does, trust me.

"Infix to Postfix" error

Usually indicates bad syntax. Can be caused by bad set commands using
the "fix" arrow:

set somevar to ActorID-\textgreater GetHealth

should be changed to:

set somevar to ( ActorID-\textgreater GetHealth )

An alternative is again a forgotten variable declaration of variable
type functions, e.g. of OnPCEquip, etc. (Thanks Horatio and Ragnar\_GD)

\hypertarget{aitravel-command-does-not-work}{%
\subparagraph{AITravel command does not
work}\label{aitravel-command-does-not-work}}

A common cause is coordinates erroneously set too far away (e.g. by
omitting a "-" or having a digit too much in there). I once had a very
nasty one of having typed two "-" which looked like one just slightly
longer than usual "-" in the editor.

\textbf{Doubled Objects\\
}There are two reasons why you may get doubled items in your game

\begin{enumerate}
\def\labelenumi{\arabic{enumi}.}
\item
  You changed your mod load order (or added a mod in the middle of the
  load order) and then continued on from a save. This is very easy to do
  accidentally by just saving a mod in the CS, which will move the mod
  in the load order.
\item
  You altered a object that had information about it stored in the save
  or Added/removed a object that is in the same cell as a object with
  information stored about it in a save.
\end{enumerate}

The best way to prevent doubling when testing work in progress mods is
to always test a mod on a save that doesn't have the mod loaded or even
better a new game.

Wrye has notes on these in much greater detail
\url{http://wrye.ufrealms.net/Wrye\%20Matching.html}

\hypertarget{script-fails-to-start-or-stops-unexpectedly}{%
\subparagraph{Script fails to start, or stops
unexpectedly}\label{script-fails-to-start-or-stops-unexpectedly}}

This can happen when a variable is not declared but the script was
nevertheless successfully saved, for example if you deleted a variable
declaration or changed the variable name after saving and didn't update
it correctly in the script. It can also be caused by the same thing in
another (different) script.

\hypertarget{crash-to-desktop-when-executing-the-script}{%
\subparagraph{Crash to desktop when executing the
script}\label{crash-to-desktop-when-executing-the-script}}

There are unfortunately a great number of possible causes. Many are
connected with not having "do once" conditions (e.g. calling certain
functions every frame). Other known problems: Removing objects with
running scripts from within their own script. Using Equip on anything
but potions (fixed with Tribunal). Casting targeted spells from an
activator (fixed with Tribunal). Using AIActivate on teleport doors that
lead outside of the active cell. Trying to use PlaceItem with the same
Object ID the script is running on. Using SetDelete on a non-disabled
object or an object with a targeted script running on it.

\hypertarget{crash-to-desktop-ctd-upon-loading-the-plugin}{%
\subparagraph{\texorpdfstring{Crash to Desktop (CTD) upon loading the
plugin
}{Crash to Desktop (CTD) upon loading the plugin }}\label{crash-to-desktop-ctd-upon-loading-the-plugin}}

One reported reason is overlong calculations: there appears to be an
issue with very long additions (e.g. adding up more than 20 variables in
one line of code) that causes a mod to CTD upon loading. If this
happens, split the calculations to several lines.